%
% OU Tikz figure styles.
%


% A reminder about tikz coordinates:
% 
% cartesian: (1,2) = "one times current x-vector, two times current y-vector"
%            (10pt,2cm) using units, but gets affected by "scale" settings
% polar: (angle in degrees : distance)
%
% Incremental:
% ++(1,2) move by (1,2) from previous positn, and make new position current


\documentclass[11pt,a4paper]{article}

\usepackage{outikz} % include the package, and OU style definitions

%
% page setup
%

\setlength{\textwidth}   {150mm}
\setlength{\textheight}  {270mm}
\setlength{\topmargin}     {-15mm}
\setlength{\headheight}    {0mm}
\setlength{\headsep}       {0mm}
\setlength{\oddsidemargin} {-10mm}
\setlength{\evensidemargin}{-10mm}

%
% Visual formatting
%

\setlength{\parskip}{2.5ex}   % space between paragraphs  was 10pt
\setlength{\parindent}{0em}
\pagestyle{plain}



\begin{document}




\section*{OU Tikz styles}

As defined in \texttt{outikz.sty}

\subsection*{Lines}
\vspace*{-2\baselineskip}

\begin{tikzpicture}

\draw(14,1) node{Dashed versions};

\draw (0,0) node[right]{Default (1pt) \verb|\draw|};
\draw (8,0) -- (10,0);
\draw (10.5,0)  node[right]{\verb|\draw[dashed]|};
\draw[dashed] (15,0) -- (17,0);


\draw (0,-1) node[right]{Thick (1.6pt) \verb|\draw[thick]|};
\draw[thick] (8,-1) -- (10,-1);
\draw (10.5,-1)  node[right]{\verb|\draw[thick,dashed]|};
\draw[thick, dashed] (15,-1) -- ++(2,0);


\draw (0,-2) node[right]{Construction lines (0.25pt) \verb|\draw[thin]|};
\draw[thin] (8,-2) -- (10,-2);
\draw (10.5,-2)  node[right]{\verb|\draw[thin,dashed]|};
\draw[thin, dashed] (15,-2) -- (17,-2);

\draw (0,-3) node[right]{Grid lines (0.5pt, 20\% black) \verb|\draw[grid]|};
\draw[grid] (8,-3) -- (10,-3);

\draw (0,-4) node[right]{Arc (0.25pt) \verb|\draw[arc]|};
\draw[arc] (8,-3.5) arc(90:180:1);

\end{tikzpicture}



\subsection*{Lines with arrows}

\begin{tikzpicture}
\draw (0,0) node[right]{Axis \verb|\draw[axis]|};
\draw[axis] (9,0) --++ (4,0);

\draw (0,-1) node[right]{Axis with ticks (0.5pt by 3pt)  and label};
\draw (0,-1.5) node[right] {\verb|\draw[axis] ... node[xlab]{$x$}|};
\draw (0,-2) node[right] {Tick need to be fixed length, independent of scaling:};
\draw (0,-2.5) node[right]  {\verb|\coordinate (t1) at (x,y);|};
\draw (0,-3) node[right]  {\verb|\draw[ticks, reset cm] (t1)--++(0,-\ticklength)|};
\draw[axis] (9,-1) --++(4,0) node[xlab] {$x$};
\coordinate (t1) at (9,-1);
\coordinate (t2) at (10,-1);
\coordinate (t3) at (11,-1);
\coordinate (t4) at (12,-1);

\draw[ticks, reset cm] (t1) --++(0,-\ticklength);
\draw[ticks, reset cm] (t2) --++(0,-\ticklength);
\draw[ticks, reset cm] (t3) --++(0,-\ticklength);
\draw[ticks, reset cm] (t4) --++(0,-\ticklength);


\draw (0,-4) node[right]{Thick axis \verb|\draw[thickaxis]|};
\draw (0,-4.5) node[right]{(for screencasts etc only)};
\draw[thickaxis] (9,-4) --++ (4,0);

\draw (0,-6) node[right]{Arc \verb|\draw[arc,-arcarrow]|};
\draw[arc,-arcarrow] (9,-6.5) arc(180:90:1);

\draw(0,-7)  node[right]{Vector \verb|\draw[vector]|};
\draw[vector] (9,-7) --++ (2,0);

\draw(0,-8)  node[right]{Force \verb|\draw[force]|};
\draw[force] (9,-8) --++ (2,0);

\draw(0,-9)  node[right]{Acceleration \verb|\draw[acceleration]|};
\draw[acceleration] (9,-9) --++ (2,0);


\draw(0,-10)  node[right]{Mapping \verb|\draw[mapping]|};
\draw[mapping] (9,-10) --++ (2,0);

\draw(0,-11)  node[right]{Network \verb|\draw[network]|};
\draw[network] (9,-11) --++ (2,0);

\draw(0,-12)  node[right]{Dimension \verb|\draw[dimension]|};
\draw[dimension] (9,-12) --++ (2,0);

\draw(0,-13)  node[right]{Dimension with markers \verb|\draw[dimensionmark]|};
\draw[dimensionmark] (9,-13) --++ (2,0);



\draw (0,-14) node[right]{Compass North \verb|\draw[-compassarrow]|};
\draw[-compassarrow] (9,-14.5) --++(0,1);
\end{tikzpicture}




\subsection*{Miscellaneous}
\vspace*{-1em}

\begin{tikzpicture}

\draw(0,0) node[right]{Point marker};
\draw(0,-0.5) node[right]{\verb|\draw (0,0) node[point, label=right:{$P$}]{};| };
\draw (14,0) node[point, label=right:{$P$}]{};


\draw(0,-1.5) node[right]{Open/closed points};
\draw(0,-2) node[right]{\verb|\draw (0,0) node[open]{} -- (1,0) node[closed]{};| };
\draw (14,-1.5) node[open]{} -- ++(1,0) node[closed]{};



\draw(0,-3) node[right]{Tag/label};
\draw(0,-3.5) node[right]
{\verb|\draw[thin] (0,0) -- (1,0) node[tag, fill=M337bluefill, anchor=west]{A label};| };
\draw[thin] (14,-3) --++ (2,0) node[tag, fill=M337bluefill, anchor=west]{A label};


\draw(0,-4.5) node[right]{Cloud};
\draw(0,-5) node[right]
{\verb|\draw (1,0) node[oucloud, fill=M337bluefill](cloudname){A label};| };
\draw(0,-5.5) node[right]
{\verb|\draw[thin] (cloudname) edge (0,0);|};
\draw (17,-4.5) node[oucloud, fill=M337bluefill](cloudname){A label};
\draw[thin] (cloudname) edge (14,-4.5);


\draw(0,-6.5) node[right]{Spring};
\draw(0,-7) node[right]
{\verb|\draw[MST210spring] (0,0) -- node[above=4] {$k,l_0$} (2,0) node[point] {\phantom{a}}|};
\draw[MST210spring] (14,-6.4) -- node[above=4] {$k,l_0$} (17,-6.4) node[point] {\phantom{a}};


\draw(0,-8) node[right]{Damper};
\draw(0,-8.5) node[right]
{\verb|\draw[MST210damper] (0,0) -- node[above=4] {$r$} (2,0)|};
\draw[MST210damper] (14,-8) -- node[above=4] {$r$} (17,-8);


\end{tikzpicture}

\subsection*{Colours}
\vspace*{-1em}


\subsubsection*{M337 line colours (for example)}
\vspace*{-1em}

\begin{tikzpicture}

\draw(10,1) node{line};
\draw(13,1) node{fill 100\%};
\draw(16,1) node{fill 40\%};

\draw (0,0) node[right]{Blue};
\draw (0,-0.5) node[right]{\verb|\draw[M337blue]|};
\draw (0,-1.0) node[right]{\verb|\draw[M337blue,fill=M337blue]|};
\draw (0,-1.5) node[right]{\verb|\draw[M337bluefill,fill=M337bluefill]|};

\draw[M337blue] (9,-.5) --++ (2,0);
\draw[M337blue, fill=M337blue] (12,0) rectangle ++(2,-1);
\draw[M337bluefill, fill=M337bluefill] (15,0) rectangle ++(2,-1);



\draw (0,-2.5) node[right]{Red};
\draw (0,-3.0) node[right]{\verb|\draw[M337red]|};
\draw (0,-3.5) node[right]{\verb|\draw[M337red,fill=M337red]|};
\draw (0,-4.0) node[right]{\verb|\draw[M337redfill,fill=M337redfill]|};

\draw[M337red] (9,-3) --++ (2,0);
\draw[M337red, fill=M337red] (12,-2.5) rectangle ++(2,-1);
\draw[M337redfill, fill=M337redfill] (15,-2.5) rectangle ++(2,-1);


\draw (0,-5) node[right]{Green};
\draw (0,-5.5) node[right]{\verb|\draw[M337green]|};
\draw (0,-6.0) node[right]{\verb|\draw[M337green,fill=M337green]|};
\draw (0,-6.5) node[right]{\verb|\draw[M337greenfill,fill=M337greenfill]|};

\draw[M337green] (9,-5.5) --++ (2,0);
\draw[M337green, fill=M337green] (12,-5) rectangle ++(2,-1);
\draw[M337greenfill, fill=M337greenfill] (15,-5) rectangle ++(2,-1);




\draw (0,-7.5) node[right]{Purple};
\draw (0,-8) node[right]{\verb|\draw[M337purple]|};
\draw (0,-8.5) node[right]{\verb|\draw[M337purple,fill=M337purple]|};
\draw (0,-9) node[right]{\verb|\draw[M337purplefill,fill=M337purplefill]|};

\draw[M337purple] (9,-8) --++ (2,0);
\draw[M337purple, fill=M337purple] (12,-7.5) rectangle ++(2,-1);
\draw[M337purplefill, fill=M337purplefill] (15,-7.5) rectangle ++(2,-1);



\draw (0,-10) node[right]{Orange};
\draw (0,-10.5) node[right]{\verb|\draw[M337orange]|};
\draw (0,-11.0) node[right]{\verb|\draw[M337orange,fill=M337orange]|};
\draw (0,-11.5) node[right]{\verb|\draw[M337orangefill,fill=M337orangefill]|};

\draw[M337orange] (9,-10.5) --++ (2,0);
\draw[M337orange, fill=M337orange] (12,-10) rectangle ++(2,-1);
\draw[M337orangefill, fill=M337orangefill] (15,-10) rectangle ++(2,-1);



\end{tikzpicture}




\subsubsection*{Structured content colours}

\begin{tikzpicture}

\draw(0,0) node[above right]{\texttt{figurebox}:};
\draw[figurebox, fill=figurebox] (5,0+1) rectangle ++(3,-2);

\draw(0,0) node[below right] {(for figure backgrounds) };

\draw(0,-3) node[right]{\texttt{buffbox} };
\draw[buffbox, fill=buffbox] (5,-3+1) rectangle ++(3,-2);

\end{tikzpicture}\hspace*{2cm}\begin{tikzpicture}

\draw(0,-6) node[right]{\texttt{greenbox} };
\draw[greenbox, fill=greenbox] (4,-6+1) rectangle ++(3,-2);

\draw(0,-9) node[right]{\texttt{bluebox} };
\draw[bluebox, fill=bluebox] (4,-9+1) rectangle ++(3,-2);


\end{tikzpicture}





\subsection*{Figure containing boxes}

(Drawn with buff background to improve visability!)\\
Of course, individuals may have their own preferred way of achieving the same effects.

\subsubsection*{Single figures}
\begin{verbatim}
\begin{outikzfig}[<colour>]{<width>}     <colour> = one of figurebox/bluebox/buffbox/greenbox 
\begin{tikzpicture}                      <width>  = one of figure/margin/solution/fullwidth
...
\end{tikzpicture}
\end{outikzfig}
\end{verbatim}


\begin{outikzfig}[buffbox]{figure}
\begin{tikzpicture}
\draw (0,0) node{Standard figure width = 360pt};
\end{tikzpicture}
\end{outikzfig}


\begin{outikzfig}[buffbox]{solution}
\begin{tikzpicture}
\draw (0,0) node{Solution figure width = 240pt};
\end{tikzpicture}
\end{outikzfig}
\hfill
\begin{outikzfig}[buffbox]{margin}
\begin{tikzpicture}
\draw (0,0) node{Margin figure width = 144pt};
\end{tikzpicture}
\end{outikzfig}

\begin{outikzfig}[buffbox]{fullwidth}
\begin{tikzpicture}
\draw (0,0) node{Full width figure = 516pt};
\end{tikzpicture}
\end{outikzfig}

\subsubsection*{Multiple figures}

\begin{verbatim}
\begin{outikzmultifig}[<colour>]{<width>}{<number>}  % <colour> = as above
                                                     % <width>  = as above
                                                     % <number> = number columns
\ousubfig                             % use \ousubfignum to number the figures
\begin{tikzpicture}                      
...
\end{tikzpicture}

\ousubfig
\begin{tikzpicture}                      
...
\end{tikzpicture}

...
\end{outikzmultifig}
\end{verbatim}



\begin{outikzmultifig}[buffbox]{figure}{3}

\ousubfig
\begin{tikzpicture}[scale=0.9]
\draw[axis] (0,0) --++(3,0) node[xlab]{$x$};
\draw[axis] (0,0) --++(0,3) node[ylab]{$y$};
\draw[domain=0:3] plot (\x,\x);
\end{tikzpicture}


\ousubfig 
\begin{tikzpicture}[scale=0.9]
\draw[axis] (0,0) --++(3,0) node[xlab]{$x$};
\draw[axis] (0,0) --++(0,3) node[ylab]{$y$};
\draw[domain=0:{sqrt(3)}] plot (\x,\x*\x);
\end{tikzpicture}

\ousubfig
\begin{tikzpicture}[scale=0.9]
\draw[axis] (0,0) --++(3,0) node[xlab]{$x$};
\draw[axis] (0,0) --++(0,3) node[ylab]{$y$};
\draw[domain=0.4:3] plot (\x, 1/\x);
\end{tikzpicture}


\ousubfig 
\begin{tikzpicture}[scale=0.9]
\draw[axis] (0,0) --++(3,0) node[xlab]{$x$};
\draw[axis] (0,0) --++(0,3) node[ylab]{$y$};
\draw[domain=0:2.9] plot (\x, 3-\x);
\end{tikzpicture}


\ousubfig 
\begin{tikzpicture}[scale=0.9]
\draw[axis] (0,0) --++(3,0) node[xlab]{$x$};
\draw[axis] (0,0) --++(0,3) node[ylab]{$y$};
\draw[domain=0:{sqrt(2.5)}] plot (\x, 2.8-\x*\x);
\end{tikzpicture}



\ousubfig 
\begin{tikzpicture}[scale=0.9]
\draw[axis] (0,0) --++(3,0) node[xlab]{$x$};
\draw[axis] (0,0) --++(0,3) node[ylab]{$y$};
\draw[domain=0.4:3] plot (\x, 3-1/\x);
\end{tikzpicture}


\ousubfig 
\begin{tikzpicture}[scale=0.9]
\draw[axis] (0,0) --++(3,0) node[xlab]{$x$};
\draw[axis] (0,0) --++(0,3) node[ylab]{$y$};
\draw[domain=0:3] plot (\x, {1.5+1.5*sin(\x*180)} );
\end{tikzpicture}


\end{outikzmultifig}




\begin{outikzmultifig}[buffbox]{figure}{2}

\ousubfignum
\begin{tikzpicture}[xscale=1.5]
\draw[axis] (0,0) --++(3,0) node[xlab]{$x$};
\draw[axis] (0,0) --++(0,3) node[ylab]{$y$};
\draw[domain=0:3] plot (\x,\x);
\end{tikzpicture}

\ousubfignum 
\begin{tikzpicture}[xscale=1.5]
\draw[axis] (0,0) --++(3,0) node[xlab]{$x$};
\draw[axis] (0,0) --++(0,3) node[ylab]{$y$};
\draw[domain=0:{sqrt(3)}] plot (\x,\x*\x);
\end{tikzpicture}



\ousubfignum
\begin{tikzpicture}[xscale=1.5]
\draw[axis] (0,0) --++(3,0) node[xlab]{$x$};
\draw[axis] (0,0) --++(0,3) node[ylab]{$y$};
\draw[domain=0:3] plot (\x, {1.5+1.5*sin(\x*180)} );
\end{tikzpicture}



\end{outikzmultifig}



\end{document}


