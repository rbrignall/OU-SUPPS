\documentclass{ouexam} 
% Available options:
% specimen			For specimen exam papers
% secondspecimen	For 2nd specimen exam
% showsolutions		Produces solution booklet

%%%%%%%%%%%%%%%%%%%%%%%%%%%%%%%%%%%%%%%%%%%
% MODULE/DATE/TIME INFORMATION
%%%%%%%%%%%%%%%%%%%%%%%%%%%%%%%%%%%%%%%%%%%
\modulecode{MU123}
\conflation{C} % Can be omitted if "specimen" or "secondspecimen" options are on
\moduletitle{Title of Module}
\examcode{MU1231606X} % Generates barcode
\copyrightyear{2016} %Optional: same as \examyear if omitted
\examyear{2016}
\exammonth{October}
\examday{Tuesday 7}
\examtime{10.00\,am -- 1.00\,pm}
\timeallowed{3 hours}
%%%%%%%%%%%%%%%%%%%%%%%%%%%%%%%%%%%%%%%%%%%%
% INSTRUCTIONS ON FIRST PAGE
%%%%%%%%%%%%%%%%%%%%%%%%%%%%%%%%%%%%%%%%%%%%
\examinstructions{%
Insert module-specific examination instructions here
}% ends \examinstructions
%%%%%%%%%%%%%%%%%%%%%%%%%%%%%%%%%%%%%%%%%%%
\begin{document}
\maketitle


%%%%%%%%%%%%%%%%%%%%%%%%%%%%%%%%%%%
% Begin writing document below here
%%%%%%%%%%%%%%%%%%%%%%%%%%%%%%%%%%%

\question[optional space for title]
The first question.\marks{12}

\begin{solution}
Solutions can be placed in-line, to generate solution files from the same source.\marks{3}

Inside solutions, notes can be added to the marks.\solnmarksplus{4}{Like this}
\end{solution}

\question
The second question.\marks{3}

\begin{solution}
Solutions can be placed in-line, to generate solution files from the same source.\marks{3}
\end{solution}

\question[(example of creating subquestions)]

\begin{enumerate}
\item Use the standard enumerate environment.\marks{3}
\item \begin{enumerate}
\item Sub-sub part.\marks{2}
\item Sub-sub part.\marks{2}
\begin{solution}
Solutions can be created per sub-question, or per sub-sub-question, or after the entire question.\marks{2}
\end{solution}
\end{enumerate}
\end{enumerate}
The enumitem package allows numbering to be resumed later, to replace the OUTeX `intertext' command.
\begin{enumerate}[resume]
\item Numbering continues from before.\marks{4}
\end{enumerate}



\question
\emph{This question is about nothing in particular.}
\begin{enumerate}
\item one\marks 4
\item two, with a sufficiently long question that it runs over more than one line.\marks{4}
\item 
\begin{enumerate}
\item Sub-sub\marks 2
\item Sub-sub that is sufficiently long that it runs over more than one line.\marks 2
\end{enumerate}
\end{enumerate}

\section{Section title}

The class file is based on the standard article class file, so commands such as section, subsection etc. can all be used if desired.



\question
\emph{This question is about nothing in particular.}
\begin{enumerate}
\item one\marks 4
\item two, with a sufficiently long question that it runs over more than one line.\marks{4}
\item 
\begin{enumerate}
\item Sub-sub\marks 2
\item Sub-sub that is sufficiently long that it runs over more than one line.\marks 2
\end{enumerate}
\end{enumerate}



\question
\emph{This question is about nothing in particular.}
\begin{enumerate}
\item one\marks 4
\item two, with a sufficiently long question that it runs over more than one line.\marks{4}
\item 
\begin{enumerate}
\item Sub-sub\marks 2
\item Sub-sub that is sufficiently long that it runs over more than one line.\marks 2
\end{enumerate}
\end{enumerate}



\question
\emph{This question is about nothing in particular.}
\begin{enumerate}
\item one\marks 4
\item two, with a sufficiently long question that it runs over more than one line.\marks{4}
\item 
\begin{enumerate}
\item Sub-sub\marks 2
\item Sub-sub that is sufficiently long that it runs over more than one line.\marks 2
\end{enumerate}
\end{enumerate}



\question
\emph{This question is about nothing in particular.}
\begin{enumerate}
\item one\marks 4
\item two, with a sufficiently long question that it runs over more than one line.\marks{4}
\item 
\begin{enumerate}
\item Sub-sub\marks 2
\item Sub-sub that is sufficiently long that it runs over more than one line.\marks 2
\end{enumerate}
\end{enumerate}

\end{document}

%% Optional section, if you want to insert solutions at the end of the document to form a mark scheme.

\solutions

Another way to generate solutions is to use the solutions switch at the end of the file. These will appear irrespective of the showsolutions option.

\question[optional space for title]
The solution to the first question.\marks{12}

\question
The solution to the second question.\marks{3}


\end{document}

